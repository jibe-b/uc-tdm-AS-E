% Copyright 2016 Institut National de la Recherche Agronomique
% 
% Licensed under the Apache License, Version 2.0 (the "License");
% you may not use this file except in compliance with the License.
% You may obtain a copy of the License at
% 
%         http://www.apache.org/licenses/LICENSE-2.0
% 
% Unless required by applicable law or agreed to in writing, software
% distributed under the License is distributed on an "AS IS" BASIS,
% WITHOUT WARRANTIES OR CONDITIONS OF ANY KIND, either express or implied.
% See the License for the specific language governing permissions and
% limitations under the License.


\documentclass[a4paper]{article}

\usepackage[latin1]{inputenc}
\usepackage{a4wide}
\usepackage{hyperref}
\usepackage{verbatim}
\usepackage[english]{babel}

\def\~{\discretionary{}{}{}} 

\include{version}
% Copyright 2016 Institut National de la Recherche Agronomique
% 
% Licensed under the Apache License, Version 2.0 (the "License");
% you may not use this file except in compliance with the License.
% You may obtain a copy of the License at
% 
%         http://www.apache.org/licenses/LICENSE-2.0
% 
% Unless required by applicable law or agreed to in writing, software
% distributed under the License is distributed on an "AS IS" BASIS,
% WITHOUT WARRANTIES OR CONDITIONS OF ANY KIND, either express or implied.
% See the License for the specific language governing permissions and
% limitations under the License.


\usepackage{float}

\newcommand{\code}[1]{
  \begin{quote}
    \texttt{#1}
  \end{quote}
}

\floatstyle{boxed}
\newfloat{program}{thp}{lop}
\floatname{program}{Figure}

\newcommand{\codefile}[3]{
  \begin{program}
    \centering
    \label{#2}
    \footnotesize
    \verbatiminput{#3}
    \normalsize
    \caption{#1}
  \end{program}
}

\newcommand{\license}{This document is part of AlvisNLP/ML software.
Copyright Institut National de la Recherche Agronomique, 2009, 2010, 2011, 2012, 2013, 2014, 2015, 2016.
}


\title{AlvisNLP Module Developer Guide\\version \version}
\author{Robert Bossy}

\begin{document}
\maketitle
\tableofcontents

\section{Copyright}
\license

\section{Introduction}
This document is a guide for developers who wish to put up additional modules types for AlvisNLP.

\subsection{Prerequisites}
AlvisNLP is written in Java 7, thus it requires a Java Virtual Machine and the JDK version 7.
It has been tested with Sun JRE XXX in 32 bit and 64 bit versions.

AlvisNLP is distributed as a Java binary (a JAR file) and Java sources under the terms of XXX licence.
The only requirements for processing corpora with the binary distribution are the requirements for the modules you chose to apply.
Indeed some modules may need external libraries or external programs.
Since the set of available modules is extensible and subject to changes, the requirements of each module should be documented separately.
The requirements of the modules supported by the BibliomeModuleFactory are specified in AlvisNLP user guide.

\section{Quick tour of the document API}
\label{DocumentAPI}

In order to write a proper AlvisNLP module, you must be familiar with the AlvisNLP document API.
Indeed your modules will most probably create documents, read or create documents metadata and sections, read or create section layers and annotations.
The reference documentation of the document API can be found as a Javadoc for the package \texttt{alvisnlp.document}.

Objects manipulated by modules are instances of the following classes:

\subsection{Corpus}
\texttt{Corpus} objects represent a corpus.
AlvisNLP modules process one corpus at a time.
The corpus to be processed will be passed to the module at runtime so there is probably little need to instanciate this class.

\subsubsection{Useful methods}

\texttt{Collection<Document> getAllDocuments()}\\
\texttt{Document getDocument(String id)}\\
\texttt{boolean hasDocument(String id)}\\
\texttt{int numDocuments()}\\
\texttt{void removeDocument(Document doc)}\\

\texttt{Iterator<Document> documentIterator()}\\
\texttt{Iterator<Document> documentIterator(Evaluator filter)}\\
\texttt{Iterator<Section> sectionIterator(Evaluator documentIterator, Evaluator sectionIterator)}\\
\texttt{Iterator<Annotation> annotationIterator(Evaluator documentIterator, Evaluator sectionIterator, String layerName, Evaluator filter)}

\subsection{Document}
\texttt{Document} objects represent documents.
Some modules process one document at a time and need to explore the structure of each document of the processed corpus.

There is no public \texttt{Document} constructor, but a static \texttt{getDocument} factory:
\texttt{Document Document.getDocument(DocumentCreator dc, Corpus corpus, String id, Logger logger)}\\
\texttt{dc} is a document creator, usually the module instance (see XXX).
If the specified corpus contains a document with the specified identifier, then the behaviour depends on the document creator.
If there is no document with the specified identifier, then this methods creates a new empty one.
The new document is automatically added to the corpus.

\subsubsection{Useful methods}

\texttt{String getId()}\\
\texttt{Corpus getCorpus()}\\
\texttt{boolean hasSection(String name)}\\
\texttt{int numSections()}\\
\texttt{void removeSecion(Section sec)}\\

\texttt{Itarator<Annotation> annotationIterator(Evaluator sectionFilter, String layerName, Evaluator filter)}\\
\texttt{Iterator<Section> sectionIterator()}\\
\texttt{Iterator<Section> sectionIterator(String name)}\\
\texttt{Iterator<Section> sectionIterator(Evaluator filter)}\\

\subsection{Section}
Some modules process one section at a time and need to explore the contents of each section of the processed document.

The only public \texttt{Section} constructor has the following signature:
\texttt{Section(SectionCreator sc, Document doc, String name, String contents)}\\
This constructor creates a section with the specified name and contents.
The new section is added to the specified document.

\subsubsection{Useful methods}

\texttt{Document getDocument()}\\
\texttt{String getName()}\\
\texttt{int getOrder()}\\
\texttt{String getContents()}\\

\texttt{Layer getAllAnnotations()}\\
\texttt{boolean hasLayer(String name)}\\
\texttt{Layer getLayer(String name)}\\
\texttt{Layer ensureLayer(String name)}\\
\texttt{Layer getLayer(String name, Evaluator annotationFilter)}\\
\texttt{Collection<Layer> getSentences(String tokenLayerName, String sentenceLayerName)}\\

\texttt{Relation ensureRelation(RelationCreator rc, String name)}\\
\texttt{Relation getRelation(String name)}\\
\texttt{boolean hasRelation(String name)}\\
\texttt{void removeRelation(Relation rel)}\\

\subsection{Layer}
Layers are annotation containers.
Annotations in layers are ordered by start position.
Annotations with the same start position are ordered by reverse end position.
Annotations with the same start and end positions are ordered in an unspecified way, though the order is stable during the whole lifetime of the layer.

\subsubsection{Constructors}

\texttt{Layer(Section sec, String name)}\\
This constructor creates a layer with the specified name and adds it to the specified section.
If the section already contains a layer with the specified name, then an \texttt{IllegalArgumentException} is raised.

\texttt{Layer(Section sec)}\\
This creates an anonymous layer.
The new layer belongs to the specified section, however the section retains no reference to the layer.
This constructor is useful to build temporary annotation containers.

\subsubsection{Useful methods}

\texttt{String getName()}\\
\texttt{Section getSection()}\\

\texttt{Layer getAnonymousCopy()}\\

\texttt{void add(Annotation a)}\\
\texttt{Annotation get(int index)}\\
\texttt{int index(Annotation a)}\\
\texttt{boolean remove(Annotation a)}\\

\texttt{boolean hasOverlaps()}\\
\texttt{void removeOverlaps(Comparator<Annotation> comp, boolean removeEqual, boolean removeIncluded, boolean removeOverlapping)}\\
\texttt{void removeOverlaps(Comparator<Annotation> comp)}\\

\texttt{Layer after(int pos)}\\
\texttt{Layer before(int pos)}\\
\texttt{Layer between(int from, int to)}\\
\texttt{Layer between(Annotation a)}\\
\texttt{Layer including(int from, int to)}\\
\texttt{Layer including(Annotation a)}\\
\texttt{Layer overlapping(int from, int to)}\\
\texttt{Layer overlapping(Annotation a)}\\
\texttt{Layer span(int from, int to)}\\
\texttt{Layer span(Annotation a)}\\

\subsection{Annotation}
An annotation belongs to a section, it is characterized by its start and end positions in the section contents.
By convention start and end positions are zero-based.
The start position is inclusive and the end position is exclusive.
Thus the difference between the end and start positions is the length of the annotation.

\subsubsection{Constructors}

\texttt{Annotation(AnnotationCreator ac, Section sec, int start, int end)}\\
This constructor creates an annotation for the specified section with the specified start and end positions.

\texttt{Annotation(AnnotationCreator ac, Layer layer, int start, int end)}\\
This constructor creates an annotation for the section to which belongs the specified layer with the specified start and end positions.
The new annotation is added into the specified layer.

\subsubsection{Useful methods}

\texttt{int getStart()}\\
\texttt{int getEnd()}\\
\texttt{int getLength()}\\
\texttt{Section getSection()}\\
\texttt{String getForm()}\\

\texttt{boolean includes(Annotation a)}\\
\texttt{boolean overlaps(Annotation a)}\\
\texttt{boolean sameSpan(Annotation a)}\\

\subsection{Relation}
A relation belongs to a section, it is characterized by a name.
The relation name is unique withi a section.

\subsubsection{Constructor}

\texttt{Relation(RelationConstructor, Section sec, String name)}\\
This constructor creates a relation in the specified section with the specified name.
The new relation is added to the section.
If the section already contains a relation with the specified name, then a \texttt{IllegalArgumentException} is raised.

\subsubsection{Useful methods}

\texttt{String getName()}\\
\texttt{Section getSection()}\\
\texttt{int size()}\\

\texttt{Collection<Tuple> getTuples()}\\
\texttt{void removeTuple(Tuple t)}\\
\texttt{void removeTuples(Collection<Tuple> c)}\\

\subsection{Tuple}
A tuple belongs to a relation.
It contains a set of key-value pairs, where keys are strings and values are annotations.
Each key is unique and contains a single annotation value.
The key is also called a \emph{role}, and the value the \emph{argument}.

\subsubsection{Constructor}

\texttt{Tuple(TupleCreator tc, Relation rel)}\\
This constructor creates a tuple for the specified relation.
The new tuple is added to the specified relation.

\subsubsection{Useful methods}

\texttt{Relation getRelation()}\\
\texttt{int getArity()}\\
\texttt{boolean hasArgument(Annotation a)}\\
\texttt{boolean hasArgument(String role)}\\
\texttt{boolean hasArgument(String role, Annotation a)}\\
\texttt{Collection<String> getRoles()}\\
\texttt{Collection<Annotation> getAllArguments()}\\
\texttt{Annotation getArgument(String role)}\\
\texttt{void setArgument(String role, Annotation a)}\\

\subsection{Element}
The classes \texttt{Corpus}, \texttt{Document}, \texttt{Section}, \texttt{Annotation}, \texttt{Relation} and \texttt{Tuple} are subclasses of \texttt{Element}.
An element has a set of key/value pairs called features.
These features can be used to store attributes and metadata.

Feature keys are not unique: there may be several values for a single key.
Though values of the same key retain the order they were inserted.
By convention, if only one value is required, then the last one is used.

Elements may have a single static feature whose key and value are constant.

\begin{tabular}[hptb]{lll}
  \textbf{Element type} & \textbf{Key} & \textbf{Value}\\
  \texttt{Corpus} & - & -\\
  \texttt{Document} & \emph{id} & document identifier\\
  \texttt{Section} & \emph{name} & section name\\
  \texttt{Annotation} & \emph{form} & annotation form\\
  \texttt{Relation} & \emph{name} & relation name\\
  \texttt{Tuple} & - & -\\
\end{tabular}
It is not possible to change the value of the static feature.

\subsubsection{Useful methods}

The following methods are accessible to all subclasses.

\texttt{boolean hasFeature(String key)}\\
\texttt{boolean isFeatureless()}\\
\texttt{List<String> getFeature(String key)}\\
\texttt{Collection<String> getFeatureKeys()}\\
\texttt{String getLastFeature(String key)}\\
\texttt{void addFeature(String key, String value)}\\
\texttt{void addFeatures(Map<String,String> map)}\\
\texttt{void addMultiFeatures(Map<String,List<String>> map)}\\
\texttt{boolean removeFeature(String key)}\\
\texttt{boolean removeFeature(String key, String value)}\\


\section{Writing a module}

\subsection{HelloWord}

The source \ref{HelloWorld} shows the source of a \texttt{HelloWorld} module type.
The first obvious thing to note is that a module type is implemented by a Java class.
When AlvisNLP runs, it creates an instance the class coresponding to each module type in the plan.

\codefile{HelloWorld module}{HelloWorld}{developper-examples/src/alvisnlp/hello/HelloWorld.java}

All classes that implement a module type must:
\begin{itemize}
\item extend \texttt{alvisnlp.module.lib.CorpusModule};
\item have the \texttt{alvisnlp.module.lib.AlvisNLPModule} annotation;
\item be public;
\item have a public null constructor.
\end{itemize}

The \texttt{CorpusModule} base class defines basic methods that allow AlvisNLP to handle the module.
The \texttt{AlvisNLPModule} annotation tells the compiler that this class is a module type implementation.
If this annotation is omitted, then AlvisNLP will not be able to instanciate a module of this type.

Note that the \texttt{HelloWorld} class only implements one method: \texttt{process}.
This is the method that AlvisNLP calls when the module must process a corpus.
The first parameter is of type \texttt{ProcessingContext}, this interface allows modules to access some information about the current AlvisNLP instance and the current plan.
The most useful methods are:
\texttt{Locale getLocale()}\\
\texttt{File getRootTempDir()}\\
\texttt{void callExternal(External<T> ext)}\\

The second parameter of the \texttt{process} method is the processed corpus.
The \texttt{Corpus} interface is explained in \ref{DocumentAPI}.

\subsection{Compiling HelloWord}
Let us make our \texttt{HelloWord} module available to AlvisNLP.
The compilation of the module is done with the \texttt{javac} program provided by the JDK.

\subsubsection{CLASSPATH}
The following libraries must be available to the compiler in order to process module sources:
\begin{itemize}
\item the JDK library;
\item the AlvisNLP libraries, available in the \texttt{lib} directory of the AlvisNLP install directory, the mandatory libraries are \texttt{util.jar} and \texttt{alvisnlp.jar};
\item any other library your modules use, in the case of \texttt{HelloWorld} there isn't any.
\end{itemize}

\subsubsection{Annotation Processor}
In order to compile the Java class into a complete AlvisNLP module type, the compiler must load the annotation processors that will recognize AlvisNLP specific annotations, such as \texttt{AlvisNLPModule}.
This is specified to the Java compiler with the \texttt{-processorpath} option.
Both libraries \texttt{util.jar} and \texttt{alvisnlp.jar} must be specified to the compiler.

The compiler will generate source files, you must specify the directory where to place them with the \texttt{-s} option.
We strongly recommend to generate source files in a separate directory so generated and hand-written source files are well discrimminated.

One of the generated files is a module factory (see \ref{ModuleFactories}).
Its fully qualified name must be provided with the \texttt{-AmoduleFactoryName}.

Putting it all together, the compilation command line looks like:

\code{javac -cp \$LIBS -d build -processorpath \$LIBS -s gen-src -AmoduleFactoryName=alvisnlp.hello.HelloFactory src/alvisnlp/hello/HelloWorld.java}

This command line assumes:
\begin{itemize}
\item \texttt{\$LIBS} contains the paths to \texttt{util.jar} and \texttt{alvisnlp.jar};
\item directories \texttt{gen-src} and \texttt{build} exist, they will receive generated sources and class files respectively;
\item the \texttt{HelloWorld.java} is located in the \texttt{src/alvisnlp/hello} directory.
\end{itemize}

\subsubsection{Compilation outcome}
The compiler output looks like:
\begin{verbatim}
Note: processing module sources
Note: looking for modules for model CORPUS (1 modules)
Note:   generating gen-src/alvisnlp/hello/HelloFactory.java
Note: checking for module classes to generate
Note: checking for module documentation to generate
Note:   generating gen-src/alvisnlp/hello/HelloWorldDoc.xml
Note: scanning org.bibliome.util.service.Service
Note:   class: alvisnlp.hello.HelloFactory
Note: scanning org.bibliome.util.service.Services
Note: writing file:build/META-INF/services/alvisnlp.factory.CorpusModuleFactory
\end{verbatim}

It basically says it scanned module classes and generated the following files:
\begin{itemize}
\item \texttt{gen-src/alvisnlp/hello/HelloFactory.java}: the module factory class
\item \texttt{gen-src/alvisnlp/hello/HelloWorldDoc.xml}: a documentation skeleton for the module \texttt{HelloWorld}
\item \texttt{build/META-INF/services/alvisnlp.factory.CorpusModuleFactory}: the service specification for modules
\end{itemize}

You can look into the contents of the generated files.
Their function will be described in the next two sections.

The final step is to wrap everything in a jar file:

\code{jar cf hello.jar -C build .}

\subsection{Module factories}
\label{ModuleFactories}

AlvisNLP instanciates modules with a module factory (\texttt{alvisnlp.factory.CorpusModuleFactory}).
Each module factory can instanciate several module types.
In our example the module factory can only instanciate the \texttt{alvisnlp.hello.HelloWorld} module type, the module type name has a short form: \texttt{HelloWorld}.
The Bibliome Module Factory provided with the AlvisNLP distribution can instanciate more than fifty module types.

AlvisNLP checks for available module factories at runtime using the \texttt{java.lang.ServiceLoader} standard facility.
The JDK Service Loader requires that the module factory implementation is specified in the file \texttt{META-INF/services/alvisnlp.factory.CorpusModuleFactory} inside the jar file.
Every class specified in the class path is automatically available to AlvisNLP.

The Annotation Processor invoked during the compilation process generates the source file for a module factory as well as the \texttt{META-INF/services/alvisnlp.factory.CorpusModuleFactory}.
So compiling modules altogether is generally sufficient to make them supported by a single module factory.
The generated factory name is specified at compile time by the \texttt{-AmoduleFactoryName} option.

In our example it is enough to put the \texttt{hello.jar} file in the Java class path to make the \texttt{HelloWorld} module type available for any user.

\subsection{Using HelloWord}
In order to use the \texttt{HelloWorld} module type, we need to set the \texttt{ALVISNLP\_HOME} to the AlvisNLP install root directory (see XXX).
We also need to place the \texttt{hello.jar} file in the Java class path.
Then we can launch AlvisNLP:

\code{alvisnlp test.xml}

Where \texttt{test.xml} looks like \ref{HelloWorldTest}.

\codefile{HelloWorld test}{HelloWorldTest}{developper-examples/test.xml}

The output of the processing log should contain something like:

\begin{verbatim}
[2012-01-24 19:34:44.238][developper-examples.hello] processing
[2012-01-24 19:34:44.240][developper-examples.hello] hello world!
[2012-01-24 19:34:44.240][developper-examples.hello] done in 0 ms
\end{verbatim}

\subsection{Documenting modules}
The documentation of AlvisNLP module classes are embedded in their distribution.
For each module type, AlvisNLP expects its documentation in an XML file in the same directory as the implementation class.
The name of the file is the same as the class with the suffix \texttt{Doc.xml}.
For the \texttt{HelloWorld} module type, this file would be \texttt{HelloWorldDoc.xml} in the same directory as \texttt{HelloWorld.java}.

During the compilation, the annotation processor looks for a documentation file.
If it cannot find a documentation file, then it generates a skeleton documentation file.
A module documentation file has the following structure:
\begin{itemize}
\item[~] \texttt{<alvisnlp-doc>}
  \begin{itemize}
  \item[~] \texttt{<synopsis>}
    \begin{itemize}
    \item[$\rightarrow$] a short description of the module type
    \end{itemize}
  \item[~] \texttt{<module-doc>}
    \begin{itemize}
    \item[~] \texttt{<description>}
      \begin{itemize}
      \item[$\rightarrow$] the full description of the module type
      \end{itemize}
    \item[~] \texttt{<param-doc name="param1">}
      \begin{itemize}
      \item[$\rightarrow$] description of the effect of \texttt{param1}
      \end{itemize}
    \item[~] \texttt{<param-doc name="param2">}
      \begin{itemize}
      \item[$\rightarrow$] description of the effect of \texttt{param2}
      \end{itemize}
    \item[~] \ldots
    \end{itemize}
  \end{itemize}
\end{itemize}

Wherever contents is expected (arrows), you can use standard XHTML tags and some additional tags:
\begin{itemize}
\item \texttt{<tag name="\ldots">}: print an XML tag with the specified name in fixed-width font. This tag may contain:
  \begin{itemize}
  \item other \texttt{<tag>} tags;
  \item \texttt{<attr name"\ldots" value="\ldots"/>} to specify one of this tag's attribute;
  \item \texttt{<text>} that contains this tag's text contents.
  \end{itemize}
\item \texttt{<this/>}: displays the name of the documented module type.
\item \texttt{<param name="\ldots"/>}: displays an anchor to the documentation for the specified parameter.
\item \texttt{<module name="\ldots">}: displays an anchor to the documentation of the specified module.
\item \texttt{<converter name="\ldots">}: displays an anchor to the documentation of the converter for the specified type.
\end{itemize}

The generated file has empty synposis and description slots.
Additionally the compilers automatically creates an empty slot for each parameter.
If the documentation file already exists, the compiler checks that all parameters are documented and that all documented parameters actually exist.
In order to make the documentation available at runtime, keep in mind that:
\begin{itemize}
\item the documentation file must be in the jar file, so you'll have to copy the documentation files from the source directory to the build directory;
\item the compiler generates the documentation file in the generated source directory (\texttt{-s} compiler option), so you'll have to move it to the source directory before editing it.
\end{itemize}

\subsubsection{I18N}
AlvisNLP supports localized documentation of modules.
In order to support the documentation of a module in a specific locale, just put a file named \texttt{ModuleClassDoc\_LOCALE.xml}.
For instance, to provide a french translation of the \texttt{HelloWorld} module type, place the documentation in a file named \texttt{HelloWorldDoc\_fr.xml}.

\subsection{Parameters}
The \texttt{HelloWorld} module is not really useful since its message is constant.
Now we will use the example of \texttt{PersonalHello} (\ref{PersonalHello}) whose message can be changed by setting a parameter.

\codefile{PersonalHello module}{PersonalHello}{developper-examples/src/alvisnlp/hello/PersonalHello.java}

Note that this module has a matching pair of accessors for a property \emph{name} (\texttt{getName} and \texttt{setName}).
Also the getter has been annotated with \texttt{@Param}.
This annotation is enough for the compiler to understand that this module type accepts a parameter \emph{name} of type \texttt{String}:
\begin{itemize}
\item AlvisNLP will set the parameter value read from the plan file through the setter method;
\item the compiler will include a slot for this parameter in the generated documentation;
\item if both getter and setter methods are abstract, then the compiler will generate an implementation, however both must be either abstract or concrete.
\end{itemize}

The type of a parameter is specified by the return type of the getter method.
It must be a reference type or an array, not a primitive, neither void.

The default value of a parameter is the value returned by the getter method after the module is instanciated and before the parameter has been set to any other value.
The parameter is said to be unset when the getter method returns \texttt{null}.

The \texttt{Param} annotation accepts the following properties:

\subsubsection{mandatory}
If \texttt{mandatory} is \texttt{true}, then it is an error for the parameter to be unset in a plan.
If \texttt{mandatory} is \texttt{false}, then the parameter may be unset.
It is the responsibility of the module implementation to handle the \texttt{null} value if the parameter is unset.
By default a parameter is mandatory.

\subsubsection{publicName}
This property specifies the name of the parameter as it should appear in a plan file.
By default the public name is the name of the getter method, stripped from the \emph{get} prefix and the first letter in lower case.

\subsubsection{defaultDoc}
The \texttt{defaultDoc} property is the text inserted in the generated documentation in the parameter slot.
By default it is the empty string.

\subsubsection{fileFlow}
This property specifies how a file parameter will be used in the module.
It only applies for parameters of the following types:
\begin{itemize}
\item \texttt{java.io.File}
\item \texttt{org.bibliome.util.TextFile}
\item \texttt{java.io.File[]}
\item \texttt{org.bibliome.util.TextFile[]}
\item \texttt{alvisnlp.module.types.FileMapping}
\item \texttt{alvisnlp.module.types.TextFileMapping}
\end{itemize}

The type of this property is the enumeration \texttt{alvisnlp.module.FileFlow} with the following values:
\begin{tabular}[hptb]{lll}
  \textbf{Value}             & \textbf{Constraints}\\
  \texttt{UNKNOWN}           & none\\
  \texttt{INPUT\_FILE}       & exists, regular and readable\\
  \texttt{OUTPUT\_FILE}      & if exists: regular and writeable, otherwise: parent directory must exist, be writable and traversable\\
  \texttt{EXECUTALE\_FILE}   & exists, regular and executable\\
  \texttt{INPUT\_DIRECTORY}  & exists, directory, readable, traversable\\
  \texttt{OUTPUT\_DIRECTORY} & if exists: directory, writable and traversable, otherwise: parent directory must exist, be writable and traversable\\
\end{tabular}

The default value is \texttt{UNKNOWN}.
If the parameter is of a file type, then we strongly advise to set this property to the appropriate value.
It allows AlvisNLP to perform chacks on the parameter values before the processing starts.

\subsubsection{minIntegerValue and maxIntegerValue}
These properties specify the allowed boundaries of numeric parameters.
They apply for parameters of the following types:
\begin{itemize}
\item \texttt{Integer}
\item \texttt{Long}
\item \texttt{Number}
\item \texttt{Integer[]}
\item \texttt{Long[]}
\item \texttt{Number[]}
\end{itemize}

The default values for \texttt{minIntegerValue} and \texttt{maxIntegerValue} are \texttt{Integer.MIN\_VALUE} and \texttt{Integer.MAX\_VALUE} respectively.
If the parameter is of a numeric type, then we strongly advise to set these properties to appropriate values.
It allows AlvisNLP to perform chacks on the parameter values before the processing starts.

\subsection{Base classes}
Our examples extend the \texttt{CorpusModule} class.
In fact there are several options for different uses, all of the ase classes inherit from \texttt{CorpusModule}.

\subsubsection{\texttt{CorpusModule}}

\paragraph{Parameters}
\texttt{Expression getActive()}

\paragraph{Useful methods}
\texttt{String getId()}\\
\texttt{String getPath()}\\

\texttt{Logger getLogger(ProcessingContext<Corpus>)}\\
\texttt{File getTempDir(File)}\\
\texttt{Timer getTimer(ProcessingContext<Corpus>, String, TimerCategory, boolean)}\\

\texttt{void processingException(String)}\\
\texttt{void rethrow(Throwable)}\\

\texttt{ExpressionResolver getGlobalExpressionResolver()}\\
\texttt{ReferenceResolver<Boolean> booleanReference(ExpressionResolver, String)}\\
\texttt{ReferenceResolver<Double> doubleReference(ExpressionResolver, String)}\\
\texttt{ReferenceResolver<List<Element>> elementsReference(ExpressionResolver, String)}\\
\texttt{ReferenceResolver<Integer> intReference(ExpressionResolver, String)}\\
\texttt{ReferenceResolver<String> stringReference(ExpressionResolver, String)}\\

\subsubsection{\texttt{DocumentModule}}
Modules that inherit from \texttt{DocumentModule} process each document independently.
Their main processing is an iteration over documents.
\texttt{DocumentModule} inherits from \texttt{CorpusModule}.

\paragraph{Parameters}
\texttt{Expression getDocumentFilter()}

\paragraph{Useful methods}
\texttt{Iterator<Document> documentIterator(ProcessingContext<Corpus>, Corpus)}

\subsubsection{\texttt{SectionModule}}
Modules that inherit from \texttt{SectionModule} process each section independently.
Their main processing is an iteration over sections.
\texttt{SectionModule} inherits from \texttt{DocumentModule}.

\paragraph{Parameters}
\texttt{Expression getSectionFilter()}

\paragraph{Abstract methods}
\texttt{String[] addFeaturesToSectionFilter()}\\
\texttt{String[] addLayersToSectionFilter()}

\paragraph{Useful methods}
\texttt{Iterator<Section> sectionIterator(ProcessingContext<Corpus>, Corpus)}\\
\texttt{Iterator<Section> sectionIterator(ProcessingContext<Corpus>, Document)}

\subsubsection{\texttt{FileReaderBase}}
Modules that inherit from \texttt{FileReaderBase} read a file or files in a directory.
Usually these files are parsed and fill the corpus with new documents and sections.

\paragraph{Parameters}
\texttt{File getSourcePath()}\\
\texttt{Pattern getAcceptPattern()}\\
\texttt{Boolean getRecursive()}

\paragraph{Abstract methods}
\texttt{void processFile(ProcessingContext<Corpus>, Corpus, File)}

\subsection{Creators}
In order to create elements, a module implementation must invoke the constructors of element subclasses.
These constructors have a formal parameter of a \emph{creator} type depending on the element type:
\begin{tabular}[hptb]{ll}
  \textbf{Element type} & \textbf{Creator type}\\
  \texttt{Document}   & \texttt{DocumentCreator}\\
  \texttt{Section}    & \texttt{SectionCreator}\\
  \texttt{Annotation} & \texttt{AnnotationCreator}\\
  \texttt{Relation}   & \texttt{RelationCreator}\\
  \texttt{Tuple}      & \texttt{TupleCreator}\\
\end{tabular}

These creator types are interfaces that define a pair of getter and setter that fills constant features to created elements.
The general practice is to make the module implementation class implement the creator interface.
You may or may not implement the abstract getter and setter methods.
Indeed the getter is annotated as a module parameter, if you do not implement the accessor, then the compiler will.

\subsection{Expressions and evaluators}
One of the most pervasive and specific parameter types is \texttt{alvisnlp.document.expression.Expression}.
It allows the user to specify an expression to be evaluated into a value that depends on the current state of the corpus.

The \texttt{Expression} type is designed as an Abstract Syntax Tree (AST) node.
That means that the only information it contains is:
\begin{itemize}
\item a constructor;
\item a list of string operands, its length is called the \emph{string arity};
\item a list of expression operands, its length is called the \emph{arity}.
\end{itemize}

In order to evaluate an expression, then it must be resolved into an evaluator of type \texttt{alvisnlp.document.expression.Evaluator}.
The resolution process recognizes supported constructors and creates a corresponding evaluator.
This evaluator can be evaluated into a boolean, an integer, a string or a list of elements.
The resolution is performed by a resolver of type \texttt{alvisnlp.document.expression.ExpressionResolver}.
You can get one with the \texttt{getGlobalExpressionResolver} method.
The typical sequence of statements looks like this:

\code{
  public void process(ProcessingContext<Corpus> ctx) \{\\
    ExpressionResolver resolver = getGlobalExpressionResolver();\\
    Evaluator evaluator = resolver.resolve(resolver, ctx, expression);\\
    \ldots\\
    // evaluate to one of the supported types\\
    boolean b = evaluator.evaluateBoolean(element);\\
    int i = evaluator.evaluateInt(element);\\
    String s = evaluator.evaluateString(element);\\
    double d = evaluator.evaluateDouble(element);\\
    Iterator<Element> it = evaluator.evaluateIterator(element);\\
    List<Element> l = evaluator.evaluateList(element);\\
  \}\\
}

\subsubsection{Child resolvers}
The resolver returned by \texttt{getGlobalExpressionResolver} can be extended to support more constructors.
In fact a resolver has a parent resolver; if an expression cannot be resolved then it is resolved by the parent resolver.
The resolver returned by \texttt{getGlobalExpressionResolver} has no parent, so an unsupported constructor is an error.

\paragraph{Action resolver}
Some expressions have side effects on the corpus data structure: they create or delete elements, change features on elements, move annotations to layers, add or remove tuple arguments\ldots
These expressions are not supported by the global expression resolver.
In order to support them you have to instanciate \texttt{alvisnlp.document.expression.resolvers.ActionExpressionResolver} with the global resolver as parent.
This resolver allows to distinguish between expressions that may have side effects.

\paragraph{Reference resolver}
In some cases, it is necessary to access a variable value in an expression, this value will depend on the processing.
The \texttt{CorpusModule} provides factory methods that return resolvers that support a reference to a value.
These methods accept two parameters: the parent resolver and the constructor that will resolve to the referenced value.
The constructor is supported with both a string arity and expression arity of zero.
You can set the value of the reference with the \texttt{setValue} method of the resolver returned by the factory method.

\paragraph{Custom resolvers}
XXX

\paragraph{Library resolvers}
XXX

\subsection{Calling external programs}
AlvisNLP/ML provides facilities to handle external programs.
The processing context method \texttt{callExternal} will operate a system call using the command line, working directory and environment provided by the argument of type \texttt{alvisnlp.module.lib.External}.
This object also handles the standard output and error of the called binary.

\section{Parameter converters}
In a plan each parameter is set using an XML tag: the tag name specifies the parameter name and the contents the parameter value.
AlvisNLP/ML converts the contents of the tag to the corresponding parameter values through an instance of \texttt{alvisnlp.converters.ParamConverter}.
The supported conversions and the instanciation of converters work the same way as modules:
\begin{itemize}
\item converters are instanciated by a \texttt{alvisnlp.converters.ParamConverterFactory};
\item converter factories are discovered at runtime through Java service loading framework.
\end{itemize}

In order to support additional conversions you must:
\begin{itemize}
\item create a class that implemnts \texttt{ParamConverter};
\item annotate this class with \texttt{alvisnlp.converters.lib.Converter};
\item provide the compiler the \texttt{-AconverterFactoryName} option.
\end{itemize}

The compiler will generate a converter factory that supports all conversions annotated with \texttt{Converter}.
Moreover the compiler wil register this factory in the service loading framework.
The conversions will then be available to AlvisNLP/ML.

\subsection{Base classes}

\subsubsection{\texttt{AbstractParamConverter<T>}}

\paragraph{Abstract Methods}
\texttt{T convertTrimmed(String)}\\
\texttt{T convertXML(String)}\\

\paragraph{Useful methods}
\texttt{void cannotConvertString(String, String)}\\
\texttt{void cannotConvertXML(org.w3c.dom.Element, String)}\\

\subsubsection{\texttt{SimpleParamConverter<T>}}
\texttt{SimpleParamConverter} inherits from \texttt{AbstractParamConverter} and implements the \texttt{convertXML} method.
The XML conversion consists in the string convertion of either:
\begin{itemize}
\item the value of the \texttt{value} attribute;
\item the value of one of the alterante attributes;
\item the string contents of the tag.
\end{itemize}

\paragraph{Abstract methods}
\texttt{String[] getAlternateAttributes()}

\subsubsection{\texttt{ArrayParamConverter<T>}}
\texttt{ArrayParamConverter} inherits from \texttt{SimpleParamConverter}.
This class is the base class for conversions to array types.
String values are split with the comma (\texttt{','}) character and each element is converted using the appropriate converter.
Thus this class assumes the conversion of the component type is supported.




\end{document}


